\documentclass[12pt, a4paaper]{article}
\usepackage{graphicx} % Required for inserting images
\usepackage[margin=1.2cm]{geometry}
\usepackage{longtable}
\usepackage{array}
\usepackage{float}
\usepackage[normalem]{ulem}

\begin{document} 
\thispagestyle{empty}
\begin{center}
   \huge{Regional Institute of Education} \\

    
\huge{Mysuru}

\vfill
\includegraphics[scale=0.5]{NCERT.png}
\vfill
\LARGE{Action Research Proposal}
\\

\Huge{ \textbf{``Enhancing Grammar Instruction Engagement in English Classroom"}}\\
\vfill


\large
\textbf{Submitted by,}\\
    Pardha Saradhi P\\
    M.Ed 2$^{\text{nd}}$ Sem

\vfill
\textbf{Submitetd to,}\\
Dr. C.G. Venkatesha Murthy\\
\vfill
\normalsize
\end{center}

\newpage 
\section{Background}   
 I pursued my B.A.B.ed in REGIONAL INSTITUTE Of EDUCATION, MYSORE as part of the course we had internship in our 7th sem During my internship at the Demonstration School, Mysore,  I had the opportunity to teach Social Science to 7th-grade students and English to 8th-grade students. While teaching English grammar to the 8th-grade class, I encountered a significant dissatisfaction stemming from the students' lack of interest in the subject.  Recognizing this challenge, I sought to address the issue by formulating an action research proposal aimed at enhancing student engagement and motivation in grammar learning through contextualized and interactive instructional strategies. 

\section{Action Research Proposal}
\textbf{Enhancing Grammar Instruction Engagement in English Classroom.}
\begin{enumerate}
\item \textbf{Perception of the Problem/Dissatisfied State}
 
Being a  English teaacher, I perceive a dissatisfactory situation in my English classroom during grammar instruction during the extension phase of lesson plan , I observe a lack of interest and engagement among my eighth-grade students. 

\item \textbf{Analysis of Dissatisfaction}

Nature of Dissatisfaction: Lack of student engagement during grammar instruction.
Extent of Dissatisfaction: Widespread across the classroom, impacting 75\% of students.
Intensity of Dissatisfaction: 50\% of Students exhibit disinterest, lack of participation, and 25 \% of students had minimal retention of grammar concepts.

\item  \textbf{Probable Causes}
\begin{enumerate}
    \item Pedagogical Approach: The way grammar is taught might not resonate with students' learning styles.
    \item Fear of Grammar: Students might feel intimidated or anxious about grammar rules and concepts.
    \item Perceived Irrelevance: Students may fail to see how grammar relates to their everyday lives, leading to disinterest.
    \item Past Experiences: If they had bad times with grammar before, they might not want to try now.
\end{enumerate}
\item \textbf{Develop Propositions}
\begin{enumerate}
    \item Pedagogical Approach: The use of a deductive method in grammar instruction may contribute to student disengagement due to its inherent limitations in catering to diverse learning styles and preferences.
    \item Fear of Grammar: Grammar instruction is commonly associated with rigid rules and complex concepts, which can induce feelings of anxiety and intimidation among students. Past negative experiences or perceived difficulty may exacerbate this fear, leading to avoidance or disengagement.

    \item Perceived Irrelevance: Students may struggle to see the practical relevance of grammar in their daily lives, viewing it as disconnected from their immediate concerns and interests. This perception of irrelevance could diminish motivation and interest in engaging with grammar learning activities.
    \item Past Experiences: Negative experiences with grammar instruction in previous grades or contexts can leave lasting impressions on students. If they have encountered difficulties or received harsh criticism in the past, they may develop a reluctance or aversion towards grammar learning, affecting their current engagement levels.
\end{enumerate}

\item \textbf{Prioritization of Proposition}

Among above prepositions the 2nd  proposition addressing the perceived irrelevance of grammar is prioritized due to its significant impact on student engagement and motivation in grammar learning. When students perceive grammar as disconnected from their daily lives and lacking practical application, their interest and willingness to engage in grammar instruction diminish, and this preposition is prioritized also because it is can dominate all the other prepositions because when student thought it is irrelavant he might not show intrest even if Pedagogical Approach is good or he does not have any fear or past negative expereiences in grammer.

\item \textbf{Action Hypothesis}

``If grammar instruction is integrated with real-life examples and interactive activities that highlight its practical relevance to students' daily lives, then students' perception of grammar as irrelevant will diminish, leading to increased motivation and interest in engaging with grammar learning activities."  

\item \textbf{Planning for Action Research}
\begin{enumerate}

    \item Timeframe
    
The action research will be conducted over a period of 3 weeks, I have 4 classes a week and  in each class 15 mins is taken for grammer transaction allowing sufficient time for planning, implementation, data collection, and analysis. The timeframe will be divided into two phases: a preparatory phase of 1 week for curriculum modification and resource preparation, followed by a 2 week intervention phase for implementing the action plan and collecting evidence.

    \item Human Resources:
    \begin{enumerate}
\item Lead Researcher (me): Responsible for overseeing the action research process, implementing intervention activities, and collecting evidence.
\item Classroom Teachers: Collaborate with classroom teachers to gather insights into student needs and provide support in implementing intervention activities.
\item Students: Actively participate in grammar lessons and provide feedback on the effectiveness of intervention activities.
\end{enumerate}

    \item Materials:
    \begin{enumerate}
\item Grammar textbooks and resources aligned with the curriculum.
\item Real-life examples and scenarios for integrating into grammar lessons.
\item Interactive activities such as usage of word search, puzzles and materials to engage students in grammar learning.
\item Data collection tools, such as surveys, observation checklists, and student work samples.
\end{enumerate}

    \item Collaborators:
    \begin{enumerate}
\item Classroom Teachers: Collaborate with grade-level English teachers to align intervention activities with the curriculum and provide support in implementing strategies.
\item School Administrators: Seek support from school administrators for accessing resources and scheduling time for collaboration and data collection.
\item Peers and Colleagues: Engage in discussions and share insights with peers and colleagues to refine intervention strategies and gather diverse perspectives.
\end{enumerate}

    \item Tools and Techniques:
    \begin{enumerate}
\item Pre - and post-intervention surveys to assess changes in student attitudes and perceptions towards grammar learning.
\item Classroom observations using observation checklists to monitor student engagement and participation during grammar lessons.
\item Student work samples to evaluate improvements in grammar comprehension and application.
\item Reflective journals or logs to document personal reflections and insights throughout the action research process.
\end{enumerate}

    \item Intervention Activities
    \begin{enumerate}
\item Curriculum Modification: Revise lesson plans to incorporate real-life examples and interactive activities that emphasize the practical relevance of grammar.
\item Interactive Discussions: Facilitate discussions on how grammar concepts apply to students' daily lives and future goals.
\item Authentic Writing Tasks: Assign writing tasks that require students to apply grammar skills in real-life communication situations, such as composing emails or letters.
\item Peer Collaboration: Encourage peer collaboration and peer feedback to promote active engagement and support in grammar learning.
\item Feedback and Reflection: Provide opportunities for students to reflect on their learning experiences and provide feedback on the effectiveness of intervention activities.
\end{enumerate}

    \item Collection of Evidence
    \begin{enumerate}
\item Pre- and post-intervention surveys administered to students to assess changes in attitudes and perceptions towards grammar learning.
\item Classroom observations conducted during grammar lessons to monitor student engagement and participation.
\item Analysis of student work samples to evaluate improvements in grammar comprehension and application.
\item Reflections and insights documented through reflective journals or logs to inform adjustments to intervention activities.
\end{enumerate}
\end{enumerate}

By carefully planning and implementing these strategies, the action research will effectively address the hypothesis and contribute to enhancing student engagement and motivation in grammar learning.
\end{enumerate}
\end{document}
